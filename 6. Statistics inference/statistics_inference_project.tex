\documentclass[]{article}
\usepackage{lmodern}
\usepackage{amssymb,amsmath}
\usepackage{ifxetex,ifluatex}
\usepackage{fixltx2e} % provides \textsubscript
\ifnum 0\ifxetex 1\fi\ifluatex 1\fi=0 % if pdftex
  \usepackage[T1]{fontenc}
  \usepackage[utf8]{inputenc}
\else % if luatex or xelatex
  \ifxetex
    \usepackage{mathspec}
  \else
    \usepackage{fontspec}
  \fi
  \defaultfontfeatures{Ligatures=TeX,Scale=MatchLowercase}
\fi
% use upquote if available, for straight quotes in verbatim environments
\IfFileExists{upquote.sty}{\usepackage{upquote}}{}
% use microtype if available
\IfFileExists{microtype.sty}{%
\usepackage[]{microtype}
\UseMicrotypeSet[protrusion]{basicmath} % disable protrusion for tt fonts
}{}
\PassOptionsToPackage{hyphens}{url} % url is loaded by hyperref
\usepackage[unicode=true]{hyperref}
\hypersetup{
            pdftitle={statistics project},
            pdfauthor={Nguyen Ngoc Duy},
            pdfborder={0 0 0},
            breaklinks=true}
\urlstyle{same}  % don't use monospace font for urls
\usepackage[margin=1in]{geometry}
\usepackage{color}
\usepackage{fancyvrb}
\newcommand{\VerbBar}{|}
\newcommand{\VERB}{\Verb[commandchars=\\\{\}]}
\DefineVerbatimEnvironment{Highlighting}{Verbatim}{commandchars=\\\{\}}
% Add ',fontsize=\small' for more characters per line
\usepackage{framed}
\definecolor{shadecolor}{RGB}{248,248,248}
\newenvironment{Shaded}{\begin{snugshade}}{\end{snugshade}}
\newcommand{\KeywordTok}[1]{\textcolor[rgb]{0.13,0.29,0.53}{\textbf{#1}}}
\newcommand{\DataTypeTok}[1]{\textcolor[rgb]{0.13,0.29,0.53}{#1}}
\newcommand{\DecValTok}[1]{\textcolor[rgb]{0.00,0.00,0.81}{#1}}
\newcommand{\BaseNTok}[1]{\textcolor[rgb]{0.00,0.00,0.81}{#1}}
\newcommand{\FloatTok}[1]{\textcolor[rgb]{0.00,0.00,0.81}{#1}}
\newcommand{\ConstantTok}[1]{\textcolor[rgb]{0.00,0.00,0.00}{#1}}
\newcommand{\CharTok}[1]{\textcolor[rgb]{0.31,0.60,0.02}{#1}}
\newcommand{\SpecialCharTok}[1]{\textcolor[rgb]{0.00,0.00,0.00}{#1}}
\newcommand{\StringTok}[1]{\textcolor[rgb]{0.31,0.60,0.02}{#1}}
\newcommand{\VerbatimStringTok}[1]{\textcolor[rgb]{0.31,0.60,0.02}{#1}}
\newcommand{\SpecialStringTok}[1]{\textcolor[rgb]{0.31,0.60,0.02}{#1}}
\newcommand{\ImportTok}[1]{#1}
\newcommand{\CommentTok}[1]{\textcolor[rgb]{0.56,0.35,0.01}{\textit{#1}}}
\newcommand{\DocumentationTok}[1]{\textcolor[rgb]{0.56,0.35,0.01}{\textbf{\textit{#1}}}}
\newcommand{\AnnotationTok}[1]{\textcolor[rgb]{0.56,0.35,0.01}{\textbf{\textit{#1}}}}
\newcommand{\CommentVarTok}[1]{\textcolor[rgb]{0.56,0.35,0.01}{\textbf{\textit{#1}}}}
\newcommand{\OtherTok}[1]{\textcolor[rgb]{0.56,0.35,0.01}{#1}}
\newcommand{\FunctionTok}[1]{\textcolor[rgb]{0.00,0.00,0.00}{#1}}
\newcommand{\VariableTok}[1]{\textcolor[rgb]{0.00,0.00,0.00}{#1}}
\newcommand{\ControlFlowTok}[1]{\textcolor[rgb]{0.13,0.29,0.53}{\textbf{#1}}}
\newcommand{\OperatorTok}[1]{\textcolor[rgb]{0.81,0.36,0.00}{\textbf{#1}}}
\newcommand{\BuiltInTok}[1]{#1}
\newcommand{\ExtensionTok}[1]{#1}
\newcommand{\PreprocessorTok}[1]{\textcolor[rgb]{0.56,0.35,0.01}{\textit{#1}}}
\newcommand{\AttributeTok}[1]{\textcolor[rgb]{0.77,0.63,0.00}{#1}}
\newcommand{\RegionMarkerTok}[1]{#1}
\newcommand{\InformationTok}[1]{\textcolor[rgb]{0.56,0.35,0.01}{\textbf{\textit{#1}}}}
\newcommand{\WarningTok}[1]{\textcolor[rgb]{0.56,0.35,0.01}{\textbf{\textit{#1}}}}
\newcommand{\AlertTok}[1]{\textcolor[rgb]{0.94,0.16,0.16}{#1}}
\newcommand{\ErrorTok}[1]{\textcolor[rgb]{0.64,0.00,0.00}{\textbf{#1}}}
\newcommand{\NormalTok}[1]{#1}
\usepackage{graphicx,grffile}
\makeatletter
\def\maxwidth{\ifdim\Gin@nat@width>\linewidth\linewidth\else\Gin@nat@width\fi}
\def\maxheight{\ifdim\Gin@nat@height>\textheight\textheight\else\Gin@nat@height\fi}
\makeatother
% Scale images if necessary, so that they will not overflow the page
% margins by default, and it is still possible to overwrite the defaults
% using explicit options in \includegraphics[width, height, ...]{}
\setkeys{Gin}{width=\maxwidth,height=\maxheight,keepaspectratio}
\IfFileExists{parskip.sty}{%
\usepackage{parskip}
}{% else
\setlength{\parindent}{0pt}
\setlength{\parskip}{6pt plus 2pt minus 1pt}
}
\setlength{\emergencystretch}{3em}  % prevent overfull lines
\providecommand{\tightlist}{%
  \setlength{\itemsep}{0pt}\setlength{\parskip}{0pt}}
\setcounter{secnumdepth}{0}
% Redefines (sub)paragraphs to behave more like sections
\ifx\paragraph\undefined\else
\let\oldparagraph\paragraph
\renewcommand{\paragraph}[1]{\oldparagraph{#1}\mbox{}}
\fi
\ifx\subparagraph\undefined\else
\let\oldsubparagraph\subparagraph
\renewcommand{\subparagraph}[1]{\oldsubparagraph{#1}\mbox{}}
\fi

% set default figure placement to htbp
\makeatletter
\def\fps@figure{htbp}
\makeatother


\title{statistics project}
\author{Nguyen Ngoc Duy}
\date{}

\begin{document}
\maketitle

\subsection{Part 1: Simulation Exercise
Instructions}\label{part-1-simulation-exercise-instructions}

\subsubsection{1.1 Simulations:}\label{simulations}

\begin{Shaded}
\begin{Highlighting}[]
\CommentTok{#install.packages("tinytex")}
\KeywordTok{library}\NormalTok{(tidyverse)}
\end{Highlighting}
\end{Shaded}

\begin{verbatim}
## -- Attaching packages --------------------------------------- tidyverse 1.3.0 --
\end{verbatim}

\begin{verbatim}
## v ggplot2 3.3.3     v purrr   0.3.4
## v tibble  3.0.4     v dplyr   1.0.2
## v tidyr   1.1.2     v stringr 1.4.0
## v readr   1.4.0     v forcats 0.5.0
\end{verbatim}

\begin{verbatim}
## -- Conflicts ------------------------------------------ tidyverse_conflicts() --
## x dplyr::filter() masks stats::filter()
## x dplyr::lag()    masks stats::lag()
\end{verbatim}

\begin{Shaded}
\begin{Highlighting}[]
\NormalTok{### As a motivating example, compare the distribution of 1000 random uniforms}
\CommentTok{#hist(runif(1000))}
\CommentTok{#and the distribution of 1000 averages of 40 random uniforms}
\CommentTok{#mns = NULL}
\CommentTok{#for (i in 1 : 1000) mns = c(mns, mean(runif(40)))}
\CommentTok{#hist(mns)}
\end{Highlighting}
\end{Shaded}

Distribution of a large collection of random exponentials

\begin{Shaded}
\begin{Highlighting}[]
\NormalTok{lambda <-}\StringTok{ }\FloatTok{0.2}
\NormalTok{n <-}\StringTok{ }\DecValTok{1000}
\NormalTok{random_exponentials <-}\StringTok{ }\KeywordTok{runif}\NormalTok{(}\KeywordTok{rexp}\NormalTok{(n,lambda))}
\KeywordTok{hist}\NormalTok{(random_exponentials)}
\end{Highlighting}
\end{Shaded}

\includegraphics{statistics_inference_project_files/figure-latex/unnamed-chunk-3-1.pdf}

Distribution of a large collection of averages of 40 exponentials.

\begin{Shaded}
\begin{Highlighting}[]
\NormalTok{mns =}\StringTok{ }\OtherTok{NULL}
\NormalTok{n=}\DecValTok{40}
\ControlFlowTok{for}\NormalTok{ (i }\ControlFlowTok{in} \DecValTok{1} \OperatorTok{:}\StringTok{ }\DecValTok{1000}\NormalTok{) mns =}\StringTok{ }\KeywordTok{c}\NormalTok{(mns, }\KeywordTok{mean}\NormalTok{(}\KeywordTok{runif}\NormalTok{(}\KeywordTok{rexp}\NormalTok{(n,lambda) )))}
\NormalTok{average_exponentials <-}\StringTok{ }\NormalTok{mns}
\KeywordTok{hist}\NormalTok{(average_exponentials)}
\end{Highlighting}
\end{Shaded}

\includegraphics{statistics_inference_project_files/figure-latex/unnamed-chunk-4-1.pdf}

\begin{Shaded}
\begin{Highlighting}[]
\NormalTok{data_combine <-}\StringTok{ }\KeywordTok{cbind}\NormalTok{(random_exponentials,average_exponentials)}
\end{Highlighting}
\end{Shaded}

\begin{Shaded}
\begin{Highlighting}[]
\KeywordTok{library}\NormalTok{(reshape2)}
\end{Highlighting}
\end{Shaded}

\begin{verbatim}
## 
## Attaching package: 'reshape2'
\end{verbatim}

\begin{verbatim}
## The following object is masked from 'package:tidyr':
## 
##     smiths
\end{verbatim}

\begin{Shaded}
\begin{Highlighting}[]
\NormalTok{data_combine_melt <-}\StringTok{ }\KeywordTok{melt}\NormalTok{(data_combine)}
\KeywordTok{head}\NormalTok{(data_combine_melt)}
\end{Highlighting}
\end{Shaded}

\begin{verbatim}
##   Var1                Var2      value
## 1    1 random_exponentials 0.85924773
## 2    2 random_exponentials 0.52937212
## 3    3 random_exponentials 0.70911244
## 4    4 random_exponentials 0.91539830
## 5    5 random_exponentials 0.60795035
## 6    6 random_exponentials 0.08834371
\end{verbatim}

\subsubsection{1.2 Show the sample mean and compare it to the
theoretical mean of the
distribution.}\label{show-the-sample-mean-and-compare-it-to-the-theoretical-mean-of-the-distribution.}

\begin{Shaded}
\begin{Highlighting}[]
\NormalTok{population_mean <-}\StringTok{ }\KeywordTok{mean}\NormalTok{(random_exponentials)}
\NormalTok{sample_mean <-}\StringTok{ }\KeywordTok{mean}\NormalTok{(average_exponentials)}
\NormalTok{sample_mean}
\end{Highlighting}
\end{Shaded}

\begin{verbatim}
## [1] 0.4985258
\end{verbatim}

\begin{Shaded}
\begin{Highlighting}[]
\NormalTok{population_mean}
\end{Highlighting}
\end{Shaded}

\begin{verbatim}
## [1] 0.5005637
\end{verbatim}

Sample\_mean and population\_mean is nearly the same

\begin{Shaded}
\begin{Highlighting}[]
\NormalTok{data_combine_melt }\OperatorTok\StringTok{ }
\StringTok{  }\KeywordTok{ggplot}\NormalTok{(}\KeywordTok{aes}\NormalTok{(}\DataTypeTok{x=}\NormalTok{Var2,}\DataTypeTok{y=}\NormalTok{value)) }\OperatorTok{+}
\StringTok{  }\KeywordTok{geom_boxplot}\NormalTok{()}
\end{Highlighting}
\end{Shaded}

\includegraphics{statistics_inference_project_files/figure-latex/unnamed-chunk-8-1.pdf}

\subsubsection{1.3. Show how variable the sample is (via variance) and
compare it to the theoretical variance of the
distribution.}\label{show-how-variable-the-sample-is-via-variance-and-compare-it-to-the-theoretical-variance-of-the-distribution.}

\begin{Shaded}
\begin{Highlighting}[]
\NormalTok{population_var <-}\StringTok{ }\KeywordTok{var}\NormalTok{(random_exponentials)}
\NormalTok{sample_var <-}\StringTok{ }\KeywordTok{var}\NormalTok{(average_exponentials)}
\NormalTok{sample_var}
\end{Highlighting}
\end{Shaded}

\begin{verbatim}
## [1] 0.002027968
\end{verbatim}

\begin{Shaded}
\begin{Highlighting}[]
\NormalTok{population_var}
\end{Highlighting}
\end{Shaded}

\begin{verbatim}
## [1] 0.08701963
\end{verbatim}

Large different between sample and population variance \#\#\# 1.4. Show
that the distribution is approximately normal.

\begin{Shaded}
\begin{Highlighting}[]
\KeywordTok{qqnorm}\NormalTok{(average_exponentials)}
\KeywordTok{qqline}\NormalTok{(average_exponentials, }\DataTypeTok{col =} \DecValTok{2}\NormalTok{)}
\end{Highlighting}
\end{Shaded}

\includegraphics{statistics_inference_project_files/figure-latex/unnamed-chunk-10-1.pdf}

\begin{Shaded}
\begin{Highlighting}[]
\CommentTok{# Test normal distribution}
\KeywordTok{shapiro.test}\NormalTok{(average_exponentials)}
\end{Highlighting}
\end{Shaded}

\begin{verbatim}
## 
##  Shapiro-Wilk normality test
## 
## data:  average_exponentials
## W = 0.99678, p-value = 0.03936
\end{verbatim}

From the output, the p-value \textgreater{} 0.05 implying that the
distribution of the data are not significantly different from normal
distribution. In other words, we can assume the normality.

\subsection{Part 2: Basic Inferential Data Analysis
Instructions}\label{part-2-basic-inferential-data-analysis-instructions}

\begin{Shaded}
\begin{Highlighting}[]
\CommentTok{#if(!require(devtools)) install.packages("devtools")}
\CommentTok{#devtools::install_github("kassambara/ggpubr")}
\KeywordTok{library}\NormalTok{(tidyverse)}
\KeywordTok{library}\NormalTok{(ggpubr)}
\end{Highlighting}
\end{Shaded}

\subsubsection{2.1 Load the ToothGrowth data and perform some basic
exploratory data
analyses}\label{load-the-toothgrowth-data-and-perform-some-basic-exploratory-data-analyses}

\begin{Shaded}
\begin{Highlighting}[]
\KeywordTok{data}\NormalTok{(}\StringTok{"ToothGrowth"}\NormalTok{)}
\NormalTok{data <-}\StringTok{ }\NormalTok{ToothGrowth}
\KeywordTok{head}\NormalTok{(data)}
\end{Highlighting}
\end{Shaded}

\begin{verbatim}
##    len supp dose
## 1  4.2   VC  0.5
## 2 11.5   VC  0.5
## 3  7.3   VC  0.5
## 4  5.8   VC  0.5
## 5  6.4   VC  0.5
## 6 10.0   VC  0.5
\end{verbatim}

\begin{Shaded}
\begin{Highlighting}[]
\KeywordTok{str}\NormalTok{(data)}
\end{Highlighting}
\end{Shaded}

\begin{verbatim}
## 'data.frame':    60 obs. of  3 variables:
##  $ len : num  4.2 11.5 7.3 5.8 6.4 10 11.2 11.2 5.2 7 ...
##  $ supp: Factor w/ 2 levels "OJ","VC": 2 2 2 2 2 2 2 2 2 2 ...
##  $ dose: num  0.5 0.5 0.5 0.5 0.5 0.5 0.5 0.5 0.5 0.5 ...
\end{verbatim}

The data has 60 observations and 3 variables (from the str() we get the
type of variables): 1. len (numeric) - Tooth length 2. supp (factor) -
Supplement type (VC or OJ) 3. dose (numeric) - Dose in milligrams

\subsubsection{2.2 Provide a basic summary of the
data.}\label{provide-a-basic-summary-of-the-data.}

\begin{Shaded}
\begin{Highlighting}[]
\KeywordTok{summary}\NormalTok{(data)}
\end{Highlighting}
\end{Shaded}

\begin{verbatim}
##       len        supp         dose      
##  Min.   : 4.20   OJ:30   Min.   :0.500  
##  1st Qu.:13.07   VC:30   1st Qu.:0.500  
##  Median :19.25           Median :1.000  
##  Mean   :18.81           Mean   :1.167  
##  3rd Qu.:25.27           3rd Qu.:2.000  
##  Max.   :33.90           Max.   :2.000
\end{verbatim}

\begin{Shaded}
\begin{Highlighting}[]
\NormalTok{data }\OperatorTok\StringTok{ }
\StringTok{  }\KeywordTok{ggplot}\NormalTok{(}\KeywordTok{aes}\NormalTok{(}\DataTypeTok{x=}\NormalTok{supp,}\DataTypeTok{y=}\NormalTok{len)) }\OperatorTok{+}
\StringTok{  }\KeywordTok{geom_boxplot}\NormalTok{()}
\end{Highlighting}
\end{Shaded}

\includegraphics{statistics_inference_project_files/figure-latex/unnamed-chunk-16-1.pdf}

\begin{Shaded}
\begin{Highlighting}[]
\NormalTok{data }\OperatorTok\StringTok{ }
\StringTok{  }\KeywordTok{ggplot}\NormalTok{(}\KeywordTok{aes}\NormalTok{(}\DataTypeTok{x=}\NormalTok{supp,}\DataTypeTok{y=}\NormalTok{len)) }\OperatorTok{+}
\StringTok{  }\KeywordTok{geom_boxplot}\NormalTok{(}\KeywordTok{aes}\NormalTok{(}\DataTypeTok{fill =}\NormalTok{ supp))}\OperatorTok{+}
\StringTok{  }\KeywordTok{facet_wrap}\NormalTok{(}\OperatorTok{~}\NormalTok{dose)}
\end{Highlighting}
\end{Shaded}

\includegraphics{statistics_inference_project_files/figure-latex/unnamed-chunk-17-1.pdf}

\begin{Shaded}
\begin{Highlighting}[]
\NormalTok{data }\OperatorTok\StringTok{ }
\StringTok{  }\KeywordTok{ggplot}\NormalTok{(}\KeywordTok{aes}\NormalTok{(}\DataTypeTok{x=}\NormalTok{dose,}\DataTypeTok{y=}\NormalTok{len)) }\OperatorTok{+}
\StringTok{  }\KeywordTok{geom_point}\NormalTok{()}\OperatorTok{+}
\StringTok{  }\KeywordTok{geom_smooth}\NormalTok{(}\DataTypeTok{method =}\NormalTok{ lm)}
\end{Highlighting}
\end{Shaded}

\begin{verbatim}
## `geom_smooth()` using formula 'y ~ x'
\end{verbatim}

\includegraphics{statistics_inference_project_files/figure-latex/unnamed-chunk-18-1.pdf}
positive effect of the dosage, as the dosage increases the tooth growth
increases. In the specific case of the VC, the tooth growth has a linear
relationship with dosage. The higher dossage (2.0mg) has less
improvement in tooth growth with the OJ supplement. However, the OJ
supplement generally induces more tooth growth than VC except at higher
dosage (2.0 mg).

\subsubsection{2.3 Hypothesis for the supplement OJ vs
VC}\label{hypothesis-for-the-supplement-oj-vs-vc}

Note that, unpaired two-samples t-test can be used only under certain
conditions:

\begin{itemize}
\tightlist
\item
  When the two groups of samples (A and B), being compared, are normally
  distributed. This can be checked using Shapiro-Wilk test.
\item
  When the variances of the two groups are equal. This can be checked
  using F-test.
\end{itemize}

Assumtion : Are the data from each of the 2 groups follow a normal
distribution?

\begin{Shaded}
\begin{Highlighting}[]
\NormalTok{VC_len <-}\StringTok{ }\NormalTok{data}\OperatorTok{$}\NormalTok{len[data}\OperatorTok{$}\NormalTok{supp }\OperatorTok{==}\StringTok{ 'VC'}\NormalTok{]}
\NormalTok{OJ_len <-}\StringTok{ }\NormalTok{data}\OperatorTok{$}\NormalTok{len[data}\OperatorTok{$}\NormalTok{supp }\OperatorTok{==}\StringTok{ 'OJ'}\NormalTok{]}
\KeywordTok{ggqqplot}\NormalTok{(VC_len)}
\end{Highlighting}
\end{Shaded}

\includegraphics{statistics_inference_project_files/figure-latex/unnamed-chunk-19-1.pdf}

\begin{Shaded}
\begin{Highlighting}[]
\KeywordTok{ggqqplot}\NormalTok{(OJ_len)}
\end{Highlighting}
\end{Shaded}

\includegraphics{statistics_inference_project_files/figure-latex/unnamed-chunk-19-2.pdf}

\begin{Shaded}
\begin{Highlighting}[]
\KeywordTok{shapiro.test}\NormalTok{(VC_len)}
\end{Highlighting}
\end{Shaded}

\begin{verbatim}
## 
##  Shapiro-Wilk normality test
## 
## data:  VC_len
## W = 0.96567, p-value = 0.4284
\end{verbatim}

\begin{Shaded}
\begin{Highlighting}[]
\KeywordTok{shapiro.test}\NormalTok{(OJ_len)}
\end{Highlighting}
\end{Shaded}

\begin{verbatim}
## 
##  Shapiro-Wilk normality test
## 
## data:  OJ_len
## W = 0.91784, p-value = 0.02359
\end{verbatim}

len of VC is normal but len from OJ is not normal distribution, we
should use Wilcoxon test

Let our null hypothesis to be there is no difference in tooth growth
when using the supplement OJ and VC.

OJ\_len = VC\_len

Let our alternate hypothesis to be there are more tooth growth when
using supplement OJ than VC.

OJ\_len \textgreater{} VC\_len

\begin{Shaded}
\begin{Highlighting}[]
\NormalTok{res <-}\StringTok{ }\KeywordTok{wilcox.test}\NormalTok{(OJ_len, VC_len,}\DataTypeTok{exact =} \OtherTok{FALSE}\NormalTok{,}\DataTypeTok{alternative =} \StringTok{"greater"}\NormalTok{)}
\NormalTok{res}
\end{Highlighting}
\end{Shaded}

\begin{verbatim}
## 
##  Wilcoxon rank sum test with continuity correction
## 
## data:  OJ_len and VC_len
## W = 575.5, p-value = 0.03225
## alternative hypothesis: true location shift is greater than 0
\end{verbatim}

The p-value of the test is 0.03, which is lower than the significance
level alpha = 0.05. We can conclude that OJ's median len is higher than
VC's median len

\subsubsection{2.3 Hypothesis for the
dossage}\label{hypothesis-for-the-dossage}

\begin{Shaded}
\begin{Highlighting}[]
\NormalTok{doseHalf =}\StringTok{ }\NormalTok{data}\OperatorTok{$}\NormalTok{len[data}\OperatorTok{$}\NormalTok{dose }\OperatorTok{==}\StringTok{ }\FloatTok{0.5}\NormalTok{]}
\NormalTok{doseOne =}\StringTok{ }\NormalTok{data}\OperatorTok{$}\NormalTok{len[data}\OperatorTok{$}\NormalTok{dose }\OperatorTok{==}\StringTok{ }\DecValTok{1}\NormalTok{]}
\NormalTok{doseTwo =}\StringTok{ }\NormalTok{data}\OperatorTok{$}\NormalTok{len[data}\OperatorTok{$}\NormalTok{dose }\OperatorTok{==}\StringTok{ }\DecValTok{2}\NormalTok{]}
\end{Highlighting}
\end{Shaded}

Assumtion : Are the data from each of the 2 groups follow a normal
distribution?

\begin{Shaded}
\begin{Highlighting}[]
\KeywordTok{ggqqplot}\NormalTok{(doseHalf)}
\end{Highlighting}
\end{Shaded}

\includegraphics{statistics_inference_project_files/figure-latex/unnamed-chunk-23-1.pdf}

\begin{Shaded}
\begin{Highlighting}[]
\KeywordTok{ggqqplot}\NormalTok{(doseOne)}
\end{Highlighting}
\end{Shaded}

\includegraphics{statistics_inference_project_files/figure-latex/unnamed-chunk-23-2.pdf}

\begin{Shaded}
\begin{Highlighting}[]
\KeywordTok{ggqqplot}\NormalTok{(doseTwo)}
\end{Highlighting}
\end{Shaded}

\includegraphics{statistics_inference_project_files/figure-latex/unnamed-chunk-23-3.pdf}

\begin{Shaded}
\begin{Highlighting}[]
\KeywordTok{shapiro.test}\NormalTok{(doseHalf)}
\end{Highlighting}
\end{Shaded}

\begin{verbatim}
## 
##  Shapiro-Wilk normality test
## 
## data:  doseHalf
## W = 0.94065, p-value = 0.2466
\end{verbatim}

\begin{Shaded}
\begin{Highlighting}[]
\KeywordTok{shapiro.test}\NormalTok{(doseOne)}
\end{Highlighting}
\end{Shaded}

\begin{verbatim}
## 
##  Shapiro-Wilk normality test
## 
## data:  doseOne
## W = 0.93134, p-value = 0.1639
\end{verbatim}

\begin{Shaded}
\begin{Highlighting}[]
\KeywordTok{shapiro.test}\NormalTok{(doseTwo)}
\end{Highlighting}
\end{Shaded}

\begin{verbatim}
## 
##  Shapiro-Wilk normality test
## 
## data:  doseTwo
## W = 0.97775, p-value = 0.9019
\end{verbatim}

From the output, the all p-values are greater than the significance
level 0.05 implying that the distribution of the data are not
significantly different from the normal distribution. In other words, we
can assume the normality.

Assumption Do the two populations have the same variances?

\begin{Shaded}
\begin{Highlighting}[]
\NormalTok{res.ftest <-}\StringTok{ }\KeywordTok{var.test}\NormalTok{(doseHalf,doseOne)}
\NormalTok{res.ftest}
\end{Highlighting}
\end{Shaded}

\begin{verbatim}
## 
##  F test to compare two variances
## 
## data:  doseHalf and doseOne
## F = 1.0386, num df = 19, denom df = 19, p-value = 0.9351
## alternative hypothesis: true ratio of variances is not equal to 1
## 95 percent confidence interval:
##  0.4110751 2.6238736
## sample estimates:
## ratio of variances 
##           1.038561
\end{verbatim}

The p-value of F-test is greater than the significance level alpha =
0.05. In conclusion, there is no significant difference between the
variances of the two sets of data. Therefore, we can use the classic
t-test witch assume equality of the two variances.

\begin{Shaded}
\begin{Highlighting}[]
\KeywordTok{t.test}\NormalTok{(doseHalf, doseOne, }\DataTypeTok{alternative =} \StringTok{"less"}\NormalTok{, }\DataTypeTok{paired =} \OtherTok{FALSE}\NormalTok{, }\DataTypeTok{var.equal =} \OtherTok{TRUE}\NormalTok{, }\DataTypeTok{conf.level =} \FloatTok{0.95}\NormalTok{)}
\end{Highlighting}
\end{Shaded}

\begin{verbatim}
## 
##  Two Sample t-test
## 
## data:  doseHalf and doseOne
## t = -6.4766, df = 38, p-value = 6.331e-08
## alternative hypothesis: true difference in means is less than 0
## 95 percent confidence interval:
##       -Inf -6.753344
## sample estimates:
## mean of x mean of y 
##    10.605    19.735
\end{verbatim}

As the p-value is lower than 0.05 (the default value for the tolerance
of the error alpha), then, we reject the null hypothesis. That can be
interpreted as there is almost null chances of obtain an extreme value
for the difference in mean of those dossages (doseHalf \textless{}
doseOne) on the tooth growth.

\begin{Shaded}
\begin{Highlighting}[]
\NormalTok{res.ftest <-}\StringTok{ }\KeywordTok{var.test}\NormalTok{(doseOne,doseTwo)}
\NormalTok{res.ftest}
\end{Highlighting}
\end{Shaded}

\begin{verbatim}
## 
##  F test to compare two variances
## 
## data:  doseOne and doseTwo
## F = 1.3687, num df = 19, denom df = 19, p-value = 0.5005
## alternative hypothesis: true ratio of variances is not equal to 1
## 95 percent confidence interval:
##  0.5417489 3.4579584
## sample estimates:
## ratio of variances 
##           1.368702
\end{verbatim}

The p-value of F-test is greater than the significance level alpha =
0.05. In conclusion, there is no significant difference between the
variances of the two sets of data. Therefore, we can use the classic
t-test witch assume equality of the two variances.

\begin{Shaded}
\begin{Highlighting}[]
\KeywordTok{t.test}\NormalTok{(doseOne, doseTwo, }\DataTypeTok{alternative =} \StringTok{"less"}\NormalTok{, }\DataTypeTok{paired =} \OtherTok{FALSE}\NormalTok{, }\DataTypeTok{var.equal =} \OtherTok{TRUE}\NormalTok{, }\DataTypeTok{conf.level =} \FloatTok{0.95}\NormalTok{)}
\end{Highlighting}
\end{Shaded}

\begin{verbatim}
## 
##  Two Sample t-test
## 
## data:  doseOne and doseTwo
## t = -4.9005, df = 38, p-value = 9.054e-06
## alternative hypothesis: true difference in means is less than 0
## 95 percent confidence interval:
##       -Inf -4.175196
## sample estimates:
## mean of x mean of y 
##    19.735    26.100
\end{verbatim}

As the p-value is lower than 0.05 (the default value for the tolerance
of the error alpha), then, we reject the null hypothesis. That can be
interpreted as there is almost null chances of obtain an extreme value
for the difference in mean of those dossages (doseOne \textless{}
doseTwo) on the tooth growth.

\end{document}
