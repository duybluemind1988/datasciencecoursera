\documentclass[]{article}
\usepackage{lmodern}
\usepackage{amssymb,amsmath}
\usepackage{ifxetex,ifluatex}
\usepackage{fixltx2e} % provides \textsubscript
\ifnum 0\ifxetex 1\fi\ifluatex 1\fi=0 % if pdftex
  \usepackage[T1]{fontenc}
  \usepackage[utf8]{inputenc}
\else % if luatex or xelatex
  \ifxetex
    \usepackage{mathspec}
  \else
    \usepackage{fontspec}
  \fi
  \defaultfontfeatures{Ligatures=TeX,Scale=MatchLowercase}
\fi
% use upquote if available, for straight quotes in verbatim environments
\IfFileExists{upquote.sty}{\usepackage{upquote}}{}
% use microtype if available
\IfFileExists{microtype.sty}{%
\usepackage[]{microtype}
\UseMicrotypeSet[protrusion]{basicmath} % disable protrusion for tt fonts
}{}
\PassOptionsToPackage{hyphens}{url} % url is loaded by hyperref
\usepackage[unicode=true]{hyperref}
\hypersetup{
            pdftitle={statistics project},
            pdfauthor={Nguyen Ngoc Duy},
            pdfborder={0 0 0},
            breaklinks=true}
\urlstyle{same}  % don't use monospace font for urls
\usepackage[margin=1in]{geometry}
\usepackage{color}
\usepackage{fancyvrb}
\newcommand{\VerbBar}{|}
\newcommand{\VERB}{\Verb[commandchars=\\\{\}]}
\DefineVerbatimEnvironment{Highlighting}{Verbatim}{commandchars=\\\{\}}
% Add ',fontsize=\small' for more characters per line
\usepackage{framed}
\definecolor{shadecolor}{RGB}{248,248,248}
\newenvironment{Shaded}{\begin{snugshade}}{\end{snugshade}}
\newcommand{\KeywordTok}[1]{\textcolor[rgb]{0.13,0.29,0.53}{\textbf{#1}}}
\newcommand{\DataTypeTok}[1]{\textcolor[rgb]{0.13,0.29,0.53}{#1}}
\newcommand{\DecValTok}[1]{\textcolor[rgb]{0.00,0.00,0.81}{#1}}
\newcommand{\BaseNTok}[1]{\textcolor[rgb]{0.00,0.00,0.81}{#1}}
\newcommand{\FloatTok}[1]{\textcolor[rgb]{0.00,0.00,0.81}{#1}}
\newcommand{\ConstantTok}[1]{\textcolor[rgb]{0.00,0.00,0.00}{#1}}
\newcommand{\CharTok}[1]{\textcolor[rgb]{0.31,0.60,0.02}{#1}}
\newcommand{\SpecialCharTok}[1]{\textcolor[rgb]{0.00,0.00,0.00}{#1}}
\newcommand{\StringTok}[1]{\textcolor[rgb]{0.31,0.60,0.02}{#1}}
\newcommand{\VerbatimStringTok}[1]{\textcolor[rgb]{0.31,0.60,0.02}{#1}}
\newcommand{\SpecialStringTok}[1]{\textcolor[rgb]{0.31,0.60,0.02}{#1}}
\newcommand{\ImportTok}[1]{#1}
\newcommand{\CommentTok}[1]{\textcolor[rgb]{0.56,0.35,0.01}{\textit{#1}}}
\newcommand{\DocumentationTok}[1]{\textcolor[rgb]{0.56,0.35,0.01}{\textbf{\textit{#1}}}}
\newcommand{\AnnotationTok}[1]{\textcolor[rgb]{0.56,0.35,0.01}{\textbf{\textit{#1}}}}
\newcommand{\CommentVarTok}[1]{\textcolor[rgb]{0.56,0.35,0.01}{\textbf{\textit{#1}}}}
\newcommand{\OtherTok}[1]{\textcolor[rgb]{0.56,0.35,0.01}{#1}}
\newcommand{\FunctionTok}[1]{\textcolor[rgb]{0.00,0.00,0.00}{#1}}
\newcommand{\VariableTok}[1]{\textcolor[rgb]{0.00,0.00,0.00}{#1}}
\newcommand{\ControlFlowTok}[1]{\textcolor[rgb]{0.13,0.29,0.53}{\textbf{#1}}}
\newcommand{\OperatorTok}[1]{\textcolor[rgb]{0.81,0.36,0.00}{\textbf{#1}}}
\newcommand{\BuiltInTok}[1]{#1}
\newcommand{\ExtensionTok}[1]{#1}
\newcommand{\PreprocessorTok}[1]{\textcolor[rgb]{0.56,0.35,0.01}{\textit{#1}}}
\newcommand{\AttributeTok}[1]{\textcolor[rgb]{0.77,0.63,0.00}{#1}}
\newcommand{\RegionMarkerTok}[1]{#1}
\newcommand{\InformationTok}[1]{\textcolor[rgb]{0.56,0.35,0.01}{\textbf{\textit{#1}}}}
\newcommand{\WarningTok}[1]{\textcolor[rgb]{0.56,0.35,0.01}{\textbf{\textit{#1}}}}
\newcommand{\AlertTok}[1]{\textcolor[rgb]{0.94,0.16,0.16}{#1}}
\newcommand{\ErrorTok}[1]{\textcolor[rgb]{0.64,0.00,0.00}{\textbf{#1}}}
\newcommand{\NormalTok}[1]{#1}
\usepackage{graphicx,grffile}
\makeatletter
\def\maxwidth{\ifdim\Gin@nat@width>\linewidth\linewidth\else\Gin@nat@width\fi}
\def\maxheight{\ifdim\Gin@nat@height>\textheight\textheight\else\Gin@nat@height\fi}
\makeatother
% Scale images if necessary, so that they will not overflow the page
% margins by default, and it is still possible to overwrite the defaults
% using explicit options in \includegraphics[width, height, ...]{}
\setkeys{Gin}{width=\maxwidth,height=\maxheight,keepaspectratio}
\IfFileExists{parskip.sty}{%
\usepackage{parskip}
}{% else
\setlength{\parindent}{0pt}
\setlength{\parskip}{6pt plus 2pt minus 1pt}
}
\setlength{\emergencystretch}{3em}  % prevent overfull lines
\providecommand{\tightlist}{%
  \setlength{\itemsep}{0pt}\setlength{\parskip}{0pt}}
\setcounter{secnumdepth}{0}
% Redefines (sub)paragraphs to behave more like sections
\ifx\paragraph\undefined\else
\let\oldparagraph\paragraph
\renewcommand{\paragraph}[1]{\oldparagraph{#1}\mbox{}}
\fi
\ifx\subparagraph\undefined\else
\let\oldsubparagraph\subparagraph
\renewcommand{\subparagraph}[1]{\oldsubparagraph{#1}\mbox{}}
\fi

% set default figure placement to htbp
\makeatletter
\def\fps@figure{htbp}
\makeatother


\title{statistics project}
\author{Nguyen Ngoc Duy}
\date{}

\begin{document}
\maketitle

\subsection{Part 1: Simulation Exercise
Instructions}\label{part-1-simulation-exercise-instructions}

\subsubsection{1.1 Simulations:}\label{simulations}

\begin{Shaded}
\begin{Highlighting}[]
\CommentTok{#install.packages("tinytex")}
\KeywordTok{library}\NormalTok{(tidyverse)}
\end{Highlighting}
\end{Shaded}

\begin{verbatim}
## -- Attaching packages --------------------------------------- tidyverse 1.3.0 --
\end{verbatim}

\begin{verbatim}
## v ggplot2 3.3.3     v purrr   0.3.4
## v tibble  3.0.4     v dplyr   1.0.2
## v tidyr   1.1.2     v stringr 1.4.0
## v readr   1.4.0     v forcats 0.5.0
\end{verbatim}

\begin{verbatim}
## -- Conflicts ------------------------------------------ tidyverse_conflicts() --
## x dplyr::filter() masks stats::filter()
## x dplyr::lag()    masks stats::lag()
\end{verbatim}

\begin{Shaded}
\begin{Highlighting}[]
\NormalTok{### As a motivating example, compare the distribution of 1000 random uniforms}
\CommentTok{#hist(runif(1000))}
\CommentTok{#and the distribution of 1000 averages of 40 random uniforms}
\CommentTok{#mns = NULL}
\CommentTok{#for (i in 1 : 1000) mns = c(mns, mean(runif(40)))}
\CommentTok{#hist(mns)}
\end{Highlighting}
\end{Shaded}

Distribution of a large collection of random exponentials

\begin{Shaded}
\begin{Highlighting}[]
\NormalTok{lambda <-}\StringTok{ }\FloatTok{0.2}
\NormalTok{n <-}\StringTok{ }\DecValTok{1000}
\NormalTok{random_exponentials <-}\StringTok{ }\KeywordTok{runif}\NormalTok{(}\KeywordTok{rexp}\NormalTok{(n,lambda))}
\KeywordTok{hist}\NormalTok{(random_exponentials)}
\end{Highlighting}
\end{Shaded}

\includegraphics{Part-1-Simulation-Exercise_files/figure-latex/unnamed-chunk-3-1.pdf}

Distribution of a large collection of averages of 40 exponentials.

\begin{Shaded}
\begin{Highlighting}[]
\NormalTok{mns =}\StringTok{ }\OtherTok{NULL}
\NormalTok{n=}\DecValTok{40}
\ControlFlowTok{for}\NormalTok{ (i }\ControlFlowTok{in} \DecValTok{1} \OperatorTok{:}\StringTok{ }\DecValTok{1000}\NormalTok{) mns =}\StringTok{ }\KeywordTok{c}\NormalTok{(mns, }\KeywordTok{mean}\NormalTok{(}\KeywordTok{runif}\NormalTok{(}\KeywordTok{rexp}\NormalTok{(n,lambda) )))}
\NormalTok{average_exponentials <-}\StringTok{ }\NormalTok{mns}
\KeywordTok{hist}\NormalTok{(average_exponentials)}
\end{Highlighting}
\end{Shaded}

\includegraphics{Part-1-Simulation-Exercise_files/figure-latex/unnamed-chunk-4-1.pdf}

\begin{Shaded}
\begin{Highlighting}[]
\NormalTok{data_combine <-}\StringTok{ }\KeywordTok{cbind}\NormalTok{(random_exponentials,average_exponentials)}
\end{Highlighting}
\end{Shaded}

\begin{Shaded}
\begin{Highlighting}[]
\KeywordTok{library}\NormalTok{(reshape2)}
\end{Highlighting}
\end{Shaded}

\begin{verbatim}
## 
## Attaching package: 'reshape2'
\end{verbatim}

\begin{verbatim}
## The following object is masked from 'package:tidyr':
## 
##     smiths
\end{verbatim}

\begin{Shaded}
\begin{Highlighting}[]
\NormalTok{data_combine_melt <-}\StringTok{ }\KeywordTok{melt}\NormalTok{(data_combine)}
\KeywordTok{head}\NormalTok{(data_combine_melt)}
\end{Highlighting}
\end{Shaded}

\begin{verbatim}
##   Var1                Var2      value
## 1    1 random_exponentials 0.35091401
## 2    2 random_exponentials 0.98489533
## 3    3 random_exponentials 0.01697637
## 4    4 random_exponentials 0.92227949
## 5    5 random_exponentials 0.39558419
## 6    6 random_exponentials 0.09013544
\end{verbatim}

\subsubsection{1.2 Show the sample mean and compare it to the
theoretical mean of the
distribution.}\label{show-the-sample-mean-and-compare-it-to-the-theoretical-mean-of-the-distribution.}

\begin{Shaded}
\begin{Highlighting}[]
\NormalTok{population_mean <-}\StringTok{ }\KeywordTok{mean}\NormalTok{(random_exponentials)}
\NormalTok{sample_mean <-}\StringTok{ }\KeywordTok{mean}\NormalTok{(average_exponentials)}
\NormalTok{sample_mean}
\end{Highlighting}
\end{Shaded}

\begin{verbatim}
## [1] 0.5022274
\end{verbatim}

\begin{Shaded}
\begin{Highlighting}[]
\NormalTok{population_mean}
\end{Highlighting}
\end{Shaded}

\begin{verbatim}
## [1] 0.501771
\end{verbatim}

Sample\_mean and population\_mean is nearly the same

\begin{Shaded}
\begin{Highlighting}[]
\NormalTok{data_combine_melt }\OperatorTok\StringTok{ }
\StringTok{  }\KeywordTok{ggplot}\NormalTok{(}\KeywordTok{aes}\NormalTok{(}\DataTypeTok{x=}\NormalTok{Var2,}\DataTypeTok{y=}\NormalTok{value)) }\OperatorTok{+}
\StringTok{  }\KeywordTok{geom_boxplot}\NormalTok{()}
\end{Highlighting}
\end{Shaded}

\includegraphics{Part-1-Simulation-Exercise_files/figure-latex/unnamed-chunk-8-1.pdf}

\subsubsection{1.3. Show how variable the sample is (via variance) and
compare it to the theoretical variance of the
distribution.}\label{show-how-variable-the-sample-is-via-variance-and-compare-it-to-the-theoretical-variance-of-the-distribution.}

\begin{Shaded}
\begin{Highlighting}[]
\NormalTok{population_var <-}\StringTok{ }\KeywordTok{var}\NormalTok{(random_exponentials)}
\NormalTok{sample_var <-}\StringTok{ }\KeywordTok{var}\NormalTok{(average_exponentials)}
\NormalTok{sample_var}
\end{Highlighting}
\end{Shaded}

\begin{verbatim}
## [1] 0.002139886
\end{verbatim}

\begin{Shaded}
\begin{Highlighting}[]
\NormalTok{population_var}
\end{Highlighting}
\end{Shaded}

\begin{verbatim}
## [1] 0.08359883
\end{verbatim}

Large different between sample and population variance \#\#\# 1.4. Show
that the distribution is approximately normal.

\begin{Shaded}
\begin{Highlighting}[]
\KeywordTok{qqnorm}\NormalTok{(average_exponentials)}
\KeywordTok{qqline}\NormalTok{(average_exponentials, }\DataTypeTok{col =} \DecValTok{2}\NormalTok{)}
\end{Highlighting}
\end{Shaded}

\includegraphics{Part-1-Simulation-Exercise_files/figure-latex/unnamed-chunk-10-1.pdf}

\begin{Shaded}
\begin{Highlighting}[]
\CommentTok{# Test normal distribution}
\KeywordTok{shapiro.test}\NormalTok{(average_exponentials)}
\end{Highlighting}
\end{Shaded}

\begin{verbatim}
## 
##  Shapiro-Wilk normality test
## 
## data:  average_exponentials
## W = 0.99858, p-value = 0.6108
\end{verbatim}

From the output, the p-value \textgreater{} 0.05 implying that the
distribution of the data are not significantly different from normal
distribution. In other words, we can assume the normality.

\end{document}
